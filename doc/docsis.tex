\chapter{DOCSIS links}
\label{chap:docsis}

ns-2 contains models for sending Internet traffic over cable modems using 
the Data Over Cable Service Interface Specification (DOCSIS) specification: 
http://www.cablemodem.com.  These models directly simulate DOCSIS 1.1 and 
DOCSIS 2.0 links and can be used to simulate DOCSIS 3.0 links and DOCSIS 
3.1 SC-QAM links.  Channel bonding for DOCSIS 3.x links is simulated by 
setting the link rate equal to the aggregate link rate for the bonding group.

DelayTb ({\tt Link/DelayTb}) models a DOCSIS downstream link (from CMTS to 
the cable modem).  More specifically, it models a single downstream service 
flow providing service to a single cable modem. It takes the following 
parameters:

\centerline{\begin{tabular}{rp{4in}}
	\code{rate\_} & "Maximum Sustained Traffic Rate":
                         i.e. Token bucket rate (bits/s) \\
	\code{bucket\_} & "Maximum Traffic Burst": 
                          i.e. Token bucket maximum size (bytes) \\
	\code{peakrate\_} & "Peak Traffic Rate": 
                            i.e. Peak rate token generation rate (bits/s) \\
	\code{peakbucket\_} & Peak rate token bucket maximum rate (bytes): 
                              leave at 1522 to model DOCSIS \\
\end{tabular}}

As per the DOCSIS 3.0/3.1 specifications, DelayTb uses two token buckets 
for rate shaping that will accumulate tokens according to their parameters.  
A departing packet gets the peak or normal transmission rate depending on 
the available tokens. To model DOCSIS 1.1/2.0, set {\tt peakrate\_} equal to 
the line rate.

DocsisLink ({\tt Link/DocsisLink}) models a DOCSIS upstream link (from 
cable modem to the CMTS).  More specifically, it models a single upstream 
service flow 
with best effort scheduling service.  It takes the following parameters:

\centerline{\begin{tabular}{rp{4in}}
	\code{mapint\_} & The MAP interval (seconds); typically 2ms \\
	\code{maxgrant\_} & The maximum grant size (bytes) per MAP interval \\
	\code{mgvar\_} & The variability of maximum grant size 
                         (0..100: percentage) \\
	\code{rate\_} & Token generation rate (bits/s) \\
	\code{bucket\_} & Token bucket maximum size (bytes) \\
	\code{peakrate\_} & Peak rate token generation rate (bits/s) \\
	\code{peakbucket\_} & Peak rate token bucket maximum rate (bytes) \\
\end{tabular}}

DOCSIS's upstream transmission is scheduled at a regular interval called 
"MAP interval".  Before the beginning of each MAP interval, the cable modem 
receives a grant 
for how many bytes it can send.  This byte count varies as a result of 
congestion from other users on the shared upstream link; 
{\tt maxgrant\_} and {\tt mgvar\_} are for emulating this congestion.  
The parameter {\tt maxgrant\_} is used 
to cap the average available capacity of the upstream link, and 
{\tt mgvar\_}
provides a way to simulate the variability of congestion.

The remaining DocsisLink parameters implement the DOCSIS token bucket rate 
shaping, just like DelayTb.

DocsisLink provides a delay estimate, via the 
{\tt DocsisLink::expectedDelay}
method, based on the token bucket parameters.  It answers "How long will 
new X bytes take to go through this link?"  Some Queue classes that 
implement AQM algorithms, such as CoDel-Dt and PIE use 
{\tt DocsisLink::expectedDelay}, when connected to a DocsisLink, to 
influence packet drop decisions.

\section{PIE and CoDel-DT}

PIE is an AQM scheme that has been adopted by DOCSIS 3.1 and DOCSIS 3.0 
standards for next generation cable modems and may be useful in other 
networks as well.  It has shown significant performance improvement in 
bufferbloat situations and it is simple to implement.  PIE randomly drops 
packets based on predicted queueing latency and its trending direction.  
Additional information can be found at 
http://tools.ietf.org/html/draft-pan-aqm-pie-01. and in 
https://datatracker.ietf.org/doc/draft-white-aqm-docsis-pie/

PIE ({\tt Queue/PIE}) takes the following parameters:

\centerline{\begin{tabular}{rp{4in}}
	\code{qdelay\_ref\_} & The reference level for queue latency (default to 20ms) \\
	\code{burst\_allowance\_} & If a queue is uncongested, up to burst allowance number of packets can be enqueued without incur drops (default to 100ms) \\
	\code{dq\_threshold\_} & Minimum queue size to measure egress rate \\
	\code{a\_, b\_} & control parameters to determine the drop probability. (default to be 0.25 and 2.5 for DOCSIS environment; 0.125 and 1.25 in other settings). It is recommended not change those parameters. \\
\end{tabular}}

CoDel-DT is a modified version of the "Controlled Delay" (CoDel) AQM.  
CoDel-DT utilizes a delay prediction at enqueue time and tail-drop in place 
of timestamps and head-drop.  If the delay prediction is sufficiently 
accurate, the performance can be shown to be identical to CoDel.  And, in 
certain situations, CoDel-DT is more easily implementable.

CoDel-DT ({\tt Queue/CoDelDt}) takes the following parameters:
\centerline{\begin{tabular}{rp{4in}}
	\code{target\_} & The target latency (default to 5ms) \\
	\code{interval\_} & Initial interval between packet drops (default to 100ms) \\
	\code{dq\_threshold\_} & Minimum queue size to measure egress rate \\
\end{tabular}}

Both PIE and CoDel-DT have the following command:
\centerline{\begin{tabular}{rp{4in}}
	\code{link link\_object} & Specifies the downstream link. \\
\end{tabular}}

PIE and CoDel-DT get latency estimates from {\tt DocsisLink::expectedDelay} 
if {\tt link\_object} is a DocsisLink.  Otherwise they utilize 
{\tt dq\_threshold\_} to estimate the egress rate, and calculate latency 
based on the estimated egress rate.
